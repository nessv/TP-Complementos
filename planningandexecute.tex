\documentclass{llncs}

\usepackage[utf8]{inputenc}
\usepackage{graphicx}
\usepackage{fancyhdr}
\usepackage{lipsum}
\pagestyle{plain}
%
\begin{document}
\[\title{Model driven Engineering en Cloud Computing. Mapeo sistemático de la literatura}

\author{Néstor Valdez, Monica Fatecha}
\institute{Facultad de Ciencias y Tecnología, Universidad Católica Nuestra Señora de la Asunción}
\maketitle


\section{Introducción}\label{sec:Introduction}
\section{Planeamiento del SMS}\label{sec:Planning}
En este apartado se muestran todas las tareas realizadas para la planificación. Se toma como base las guías propuestas por
Kitchenham\cite{slr}, pero se toma en cuenta que las guías están mas bien basadas en una SLR.
\subsection{Identificar la necesidad de la revisión}
Se han buscado mapeos sistemáticos en el contexto de la ingeniería dirigida por modelos (MDE) y Cloud Computing, no obteniendo
resultados, siendo esta una de las razones que nos lleva a realizar un SMS sobre el tema en cuestión.

El objetivo de este SMS es no sólo presentar trabajos existentes, sino también mostrar la proyección que tendra MDE con Cloud computing
en el futuro.
\subsection{Formular las preguntas de investigación}
Las preguntas de investigación y las motivaciones de las mismas, están definidas en la Tabla 1.

\subsection{Antecedentes}
La ingenieria dirigida por modelos (MDE) se está convirtiendo en el software dominante para especificar,
desarrollar y mantener software. En MDE, los modelos son los protagonistas principales en el proceso
de ingeniería y son usados en varios niveles implementatívos.
Al mismo tiempo, Software as a Service (Saas), está ganando popularidad como una forma estándar para el diseño
e implementación de aplicaciones de software.



\begin{tabular}{  | p{5cm} | p{7cm}  | }
  \hline
  \multicolumn{2}{|c|}{\bf Tabla 1} \\
  \hline
  \bf Pregunta & \bf Motivación \\
  \hline
  P1. Qué podría aportar?  & Descubrir qué tipo de aportes daría MDE en Cloud Computing y evaluar el resultado de forma general\\
  \hline
  P2. Actualmente que aportes brinda MDE a Cloud Computing? & Investigar cuales son los estudios o investigaciones presentes en la actualidad, analizar e identificar si se pueden extender dichas investigaciones\\
  \hline
  P3. Que iniciativas se están tomando? & Evaluar MasS. Su viabilidad y permanencia en el tiempo\\
  \hline
  P4. Qué proyección tiene la iniciativa Modeling As a Service? & Identificar cómo puede mejorar la nube con esta implementación y estudiar esas mejoras\\
  \hline
  P5. Cómo las técnicas de MDE pueden aportar mejor performance en Cloud Computing?  & Evaluar los factores que podrían afectar en la economía de la nube al implementar MDE\\
  \hline
  P6. Cómo MDE podría afectar la economía de la nube y que nuevos negocios podría atraer? & Realizar una investigación de cuáles son los beneficios que puede ofrecer MDE en la nube, en este caso si se pueden generar nuevas formas de negocio\\
  \hline
\end{tabular}

\subsection{Estrategia de Búsqueda}
Realizar la busqueda automatica  en el periodo comprendido entre 2009 y 2016 y en las siguientes fuentes: IEEE XPLORE en el area de computing and processing, SCOPUS en el area de Computer Science, Web of Science.

La seleccion realizada sobre los terminos principales, sinónimos, palabras alternativas o terminos relacionados con los terminos principales se presentan en la Tabla 2. Buscamos publicaciones en ingles ya que es el idioma universal en el campo de la investigación.\\ \\
\begin{tabular}{  | p{5cm} | p{7cm}  | }
  \hline
  \multicolumn{2}{|c|}{\bf Tabla 2} \\
  \hline
  \bf Terminos principales & \bf Terminos alternativos \\
  \hline
    MDE & mde OR Model Driven Engineering OR model driven OR model driven OR model as a service OR modeling as a service OR mass\\
   \hline
    Implementation & implementation OR implement OR implementing OR with OR aplication OR use OR utilization\\
   \hline
   Cloud Computing & cloud computing OR  cloud OR cloud c\\
   \hline
\end{tabular}\\ \\
\pagebreak

La cadena de busqueda definida a partir de la Tabla 2 es la siguiente: \\
"( mde OR Model Driven Engineering OR model driven OR model driven OR model as a service OR modeling as a service OR mass)AND( implementation OR implement OR implementing OR with OR aplication OR use OR utilization)AND( cloud computing OR  cloud OR cloud c)"

La cadena de búsqueda  se aplicara en las fuentes indicadas, buscando en el título y en el resumen, en caso de que el buscador acepte y si no buscará en el texto completo.\\ \\
Todas las decisiones tomadas durante la definición de la cadena de búsqueda se tomaron conjuntamente entre los autores.

\subsection{Criterios de selección de estudios}
Se incluiran en ingles los artículos que se refieran a la implementacion del cloud computing con MDE y publicados entre 2009 y 2016 en revistas y conferencias, congresos o talleres, tambien se tomaran los terminos relevantes que aparezcan en el abstract.\\\\
Se excluiran tipos de articulos de depate, o en forma de resumen o de presentacion, cuya contribucion no se relacione con la implementacion de MDE con cloud computing y tambien se excluiran articulos relacionados a la relacion de cloud computing con otros modelos o metodos.

\subsection{Procedimiento para la selección de estudios}
Para seleccionar los estudios primarios se aplicarán los criterios de inclusión/exclusión leyendo los artículos de revistas, de conferencia o workshops encontrados.Si tras leer el resumen todavia quedan dudas sobre la inclusión/exclusión de algun articulo se leera el artículo completo.\\ \\
La selección de estudios la realizará el primer autor del trabajo y la segunda autora seleccionará un porcentaje no mayor al 40  \% de los artículos  para verificar si los criterios de inclusión/exclusión se aplicarion correctamente.

\subsection{Lista de comprobación y procedimiento para la evaluación de la calidad de los estudios}
Como criterio para considerar artículos de cierta calidad, se consideró seleccionar estudios en coferencias, cogresos o talleres de prestigio con revisión de pares.

\subsection{Estrategia para extracción de los datos}
Para la estracción de datos, mencionamos dos partes, la primera con los datos de cada estudio: autores, tipo de publicación, etc. Y la segunda parte que contiene las propiedades del esquema definido para clasificar los estudios que se seleccionó.\\ \\

%\begin{tabular}{  | p{5cm} | p{7cm}  | }
%  \hline
%  \multicolumn{2}{|c|}{\bf Tabla 3} \\
%  \hline
%  \bf Dimensiones & \bf Categorias \\
%  \hline
%   Tipo de aportes futuros & \\
%  \hline
%   Tipo de aportes actuales & \\
%  \hline
%   Iniciativas & \\
%   \hline
%   Iniciativas en el tema MasS & \\
%   \hline
%    Tipo de tecnicas & \\
%   \hline
%    Negocios y impacto economico & \\
%    \hline
%\end{tabular}
\subsection{Síntesis de los datos extraídos}

Se realizará una síntesis cuantitativa considerando el numero de articulos en cada dimensio/categoria, mostrando a través de tablas y/o graficos, para responder a cada pregunta de investigación.
También se analizará numero de publicaciones por año y tipo de publicación para detectar y justificar tendencias y orientar a futuros investigadores sobre los foros más apropiados en los que se puede buscar información en temas relacionados con MDE con cloud computing.

\subsection{Realizar la revisión}
Este SMS se llevó a cabo en una fase, se localizaron artículos encontrados entre 2010 y 2016. La cronología del proceso para la realización de este SMS se muestra en la Tabla 4. \\
\begin{tabular}{  | p{2cm} | p{3cm}  |  p{3cm} | p{3cm} |}
  \hline
  \multicolumn{4}{|c|}{\bf Tabla 3} \\
  \hline
  \bf Fecha  & \bf Planificar & \bf Realizar & \bf Resultado \\
  \hline
   2 de Noviembre  & definición del tema & & implementación con cloud computing\\
  \hline
   7 y 8 de Noviembre &  & identificación de la literatura relevante & lista de los articulos encontrados en cada fuente de busqueda.(8 articulos)\\
   \hline
    12 de Noviembre  & & seleccion de estudios primarios leyendo resumenes & lista con datos de los articulos seleccionados (5 articulos)\\
    \hline
     15 de Noviembre & & seleccion de estudios y extraccion de datos leyendo el texto & lista de la extracción de datos. (4 articulos) \\
    \hline
 \end{tabular}



\subsection{Identificar y seleccionar los estudios primarios}
Se encontraron 15 articulos entre los años 2012 y 2016, aplicando la estrategia de búsqueda definida en el protocolo. Debido a la limitación que ofrecen ciertas fuentes de búsqueda, en el caso que no permitieran usar cadenas de búsqueda complejas, se tuvieron que crear cadenas específicas para cada fuente y manipular los resultados de las búsquedas para obtener los mismos resultados que pudieran haber sido obtenidos utilizando la cadena de búsqueda original. La búsqueda se hizo en el título.\\
Para cada fuente de búsqueda se guardaron: las cadenas de búsqueda, los metadatos de los articulos encontrados y resumenes.\\
Después de leer los resumenes de los artículos y excluir los que no tenían nada que ver con MDE con cloud computing, solo quedaron 10, no habia duplicados.A continuación se aplicaron los criterios de inclusión/exclusión a los articulos restantes, leyendo el texto completo. Se excluyeron articulos que se referian a cloud computing y la implementación con otro metodo.\\
Como se planificó en el protocolo, la identificación y la selección de estudios primarios lo realizó el primer autor del artículo, y la segunda autora escogió el 40 \% de los artículos para la corrección.Las dudas que surgieron durante la selección se resolvieron entre los dos autores.\\ \\ \\

\subsubsection{Extracción de datos}
Primero se comenzó a extraer los metadatos y clasificar los 10 articulos seleccionados, leyendo el texto completo. Para la clasificación se utilizó el esquema de clasificación presentado en la Tabla 2. Se decidió excluir 2 articulos quedando así 8 estudios primarios.

\begin{thebibliography}{1}
\bibitem{slr}
Guidelines for performing Systematic Literature Reviews in Software Engineering
Barbara Kitchenham, und Stuart Charters. EBSE 2007-001. Keele University and Durham University Joint Report, (2007)
\bibitem{ref1}
Combining Model-Driven Engineering and Cloud Computing
Hugo Bruneliere, Jordi Cabot and Frédéric Jouault AtlanMod, INRIA RBA Center and EMN
4 rue Alfred Kastler, 44307 Nantes, France
\bibitem{}
Model-Driven Engineering for High Performance and CLoud computing
Ileana Ober, Aniruddha Gokhale, James Hill, Jean-Michel Bruel1, Michael Felderer, David Lugato, and Akshay Dabholkar
\bibitem{}
MDE Opportunities in Multi-Tenant Cloud Applications
Mohammad Abu Matar and Jon Whittle
Etisalat British Telecom Innovation Center Khalifa University of Science, Technology and Research Abu Dhabi, United Arab Emirates
\bibitem{}
Multi‐Cloud Computing Platform Support
With Model‐Driven Application Runtime Framework
Nacha Chondamrongkul and Punnarumul Temdee
School of Information Technology, Mah Fah Luang University Chiang rai, Thailand
\bibitem{}
Model-driven specification of adaptive cloud-based systems
Nagel, B., Gerth, C., Yigitbas, E., Christ, F., Engels, G.
 1st International Workshop on Model-Driven Engineering for High Performance and CLoud
\bibitem{}
Modelling and comparing cloud computing service level agreements
Alkandari, F., Paige, R.F.
 1st International Workshop on Model-Driven Engineering for High Performance and CLoud
Computing, MDHPCL 2012 - Satellite Event of MODELS 2012
\bibitem{}
An Integrated Meta-model for Cloud Application Security Modelling
Kyriakos Kritikosa, Philippe Massonet
ICS-FORTH, Heraklion, Crete, Greece bCETIC, Charleroi, Belgium
\bibitem{}
Towards a Model-Driven Solution to the Vendor
Lock-in Problem in Cloud Computing
Gabriel Costa Silva (PhD Candidate), Louis M. Rose and Radu Calinescu (Supervisors)
Department of Computer Science, University of York
\bibitem{}
Evolution Feature Oriented Model Driven
Product Line Engineering Approach for
Synergistic and Dynamic Service
Evolution in Clouds:
A04BPEL3.0 Proposal
Zhe Wang, Kevin Chalmers,Liu Xiaodong
School of Computing
Edinburgh Napier University
\end{thebibliography}

\]
\end{document}