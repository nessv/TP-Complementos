% !TEX encoding = UTF-8 Unicode
\documentclass{llncs}
\usepackage[utf8]{inputenc}
\usepackage{graphicx}
\usepackage{fancyhdr}
\usepackage{lipsum}
\pagestyle{plain}
%
\begin{document}


\title{Model driven Engineering en Cloud Computing. Mapeo sistemático de la literatura}

\author{Néstor Valdez, Monica Fatecha}
\institute{Facultad de Ciencias y Tecnología, Universidad Católica Nuestra Señora de la Asunción}
\maketitle


\section{Introducción}\label{sec:Introduction}
\section{Planeamiento del SMS}\label{sec:Planning}
En este apartado se muestran todas las tareas realizadas para la planificación. Se toma como base las guias propuestas por
Kitchenham\cite{slr}, pero se toma en cuenta que las guias estas mas bien basadas en una SLR.
\subsection{Identificar la necesidad de la revisión}
Se han buscado mapeos sistematicos en el contexto de la ingenieria dirigida por modelos (MDE) y Cloud Computing, no obteniendo
resultados, siendo esta una de las razones que nos lleva a realizar un SMS sobre el tema en cuestión.

El objetivo de este SMS es no sólo presentar trabajos existentes, sino también mostrar la proyección que tendra MDE con Cloud computing
en el futuro.
\subsection{Formular las preguntas de investigación}
Las preguntas de investigacion y las motivaciones de las mismas, estan definidas en la Tabla 1.

\subsection{Antecedentes}
La ingenieria dirigida por modelos (MDE) se está convirtiendo en el software dominante para especificar,
desarrollar y mantener software. En MDE, los modelos son los protagonistas principales en el proceso
de ingenieria y son usados en varios niveles implementativos.
Al mismo tiempo, Software as a Service (Saas), está ganando popularidad como una forma estandar para el diseño
e implementación de aplicaciones de software.

\begin{tabular}{  | p{5cm} | p{7cm}  | }
  \hline
  \multicolumn{2}{|c|}{\bf Tabla 1} \\
  \hline
  \bf Pregunta & \bf Motivación \\
  \hline
  P1. Qué podría aportar?  & Descubrir qué tipo de aportes daría MDE en cloud computing y evaluar el resultado de forma general\\
  \hline
  P2. Actualmente que aportes brinda MDE a Cloud Computing? & Investigar cuales son los estudios o investigaciones presentes en la actualidad, analizar e identificar si se pueden extender dichas investigaciones\\
  \hline
  P3. Que iniciativas se estan tomando? & Evaluar MasS. Su viabilidad y permanencia en el tiempo\\
  \hline
  P4. Qué proyección tiene la iniciativa Modeling As a Service? & Identificar cómo puede mejorar la nube con esta implementación y estudiar esas mejoras\\
  \hline
  P5. Cómo las tecnicas de MDE pueden aportar mejor performance en Cloud Computing?  & Evaluar los factores que podrían afectar en la economía de la nube al implementar MDE\\
  \hline
  P6. Cómo MDE podriá afectar la economia de la nube y que nuevos negocios podría atraer? & Realizar una investigación de cuáles son los beneficios que puede ofrecer MDE en la nube, en este caso si se pueden generar nuevas formas de negocio\\
  \hline
\end{tabular}
\begin{thebibliography}{1}
\bibitem{slr}
Guidelines for performing Systematic Literature Reviews in Software Engineering
Barbara Kitchenham, und Stuart Charters. EBSE 2007-001. Keele University and Durham University Joint Report, (2007)
\bibitem{ref1}
Combining Model-Driven Engineering and Cloud Computing
Hugo Bruneliere, Jordi Cabot and Frédéric Jouault AtlanMod, INRIA RBA Center & EMN
4 rue Alfred Kastler, 44307 Nantes, France
\bibitem{}
Model-Driven Engineering for High Performance and CLoud computing
Ileana Ober, Aniruddha Gokhale, James Hill, Jean-Michel Bruel1, Michael Felderer, David Lugato, and Akshay Dabholkar
\bibitem{}
MDE Opportunities in Multi-Tenant Cloud Applications
Mohammad Abu Matar and Jon Whittle
Etisalat British Telecom Innovation Center Khalifa University of Science, Technology and Research Abu Dhabi, United Arab Emirates
\end{thebibliography}

\end{document}
